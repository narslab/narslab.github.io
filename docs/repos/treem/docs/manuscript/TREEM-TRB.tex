% Transportation Research Board conference paper template
% version 1.1
% 
% David R. Pritchard, http://davidpritchard.org
%   1.0 - Mar. 2009
%   1.1 - Sep. 2011, fixes for captions

% PAGE LAYOUT
%------------------------------------------

% Custom paper settings...
\documentclass[titlepage,oneside,12pt]{article}

\oddsidemargin 0.0in
\topmargin -0.5in
\headheight 0.3in
\headsep 0.2in
\textwidth 6.5in
\textheight 9.0in
\setlength{\parindent}{0.5in}

\usepackage{hyperref}

% PAGE HEADER
%------------------------------------------
% Adjust the header text below (INSERT AUTHORS HERE)
\oddsidemargin 0.0in
\usepackage{lscape}
\usepackage[tiny,rm]{titlesec}
\usepackage{titling}
\usepackage{xcolor}
\newpagestyle{trbstyle}{
	\sethead{Apostolov and Oke}{}{\thepage}
}
\pagestyle{trbstyle}

% HEADINGS
%------------------------------------------
\titleformat{\section}{\bfseries}{}{0pt}{\uppercase}
\titlespacing*{\section}{0pt}{12pt}{*0}
\titleformat{\subsection}{\bfseries}{}{0pt}{}
\titlespacing*{\subsection}{0pt}{12pt}{*0}
\titleformat{\subsubsection}{\itshape}{}{0pt}{}
\titlespacing*{\subsubsection}{0pt}{12pt}{*0}

% LISTS
%------------------------------------------
% Adjust lists a little. Not quite perfectly fitting TRB style, but vaguely
% close at least.
%\usepackage{enumitem}
%\setlist[1]{labelindent=0.5in,leftmargin=*}
%\setlist[2]{labelindent=0in,leftmargin=*}

\usepackage{booktabs}
% CAPTIONS
%------------------------------------------
% Get the captions right. Authors must still be careful to use "Title Case"
% for table captions, and "Sentence case." for figure captions.
\usepackage{ccaption}
\usepackage{amsmath}
\makeatletter
\renewcommand{\fnum@figure}{\textbf{FIGURE~\thefigure} }
\renewcommand{\fnum@table}{\textbf{TABLE~\thetable} }
\makeatother
\captiontitlefont{\bfseries \boldmath}
\captiondelim{\;}
%\precaption{\boldmath}

% FONTS
%------------------------------------------
% Three options for fonts. I prefer Times for text and Computer Modern for
% math.

% Times for text, Computer Modern for math
\usepackage{times}
% Times for text and math
%\usepackage{pslatex}
% Times for text and math
%\usepackage{times,mathptmx} 

% Some pdf conversion tricks? Unsure.
\usepackage[T1]{fontenc}
\usepackage{textcomp}
\usepackage{bm}
\usepackage{makecell}
\usepackage{subfig}
\def\checkmark{\tikz\fill[scale=0.4](0,.35) -- (.25,0) -- (1,.7) -- (.25,.15) -- cycle;}

% CITATIONS
%------------------------------------------
% TRB uses an Author (num) citation style. I haven't found a way to make
% LaTeX/Bibtex do this automatically using the standard \trbcite macro, but
% this modified \trbcite macro does the trick.

% TODO: sort&compress option?
\usepackage[sort,numbers]{natbib}
\newcommand{\trbcite}[1]{({\it \citenum{#1}})}
\newcommand{\trbcitet}[1]{\citeauthor{#1} ({\it \citenum{#1}})}

\setcitestyle{round}

%% Cite Title
%\usepackage[style=authoryear,backend=biber,natbib,maxcitenames=2,doi=false,isbn=false,url=false,eprint=false]{biblatex}
%\addbibresource{bib/references.bib}

%\usepackage{hyperref}

% LINE NUMBERING
%------------------------------------------
% TRB likes line numbers on drafts to help reviewers refer to parts of the
% document. Comment out for final versions.
\usepackage{lineno}
\renewcommand\linenumberfont{\normalfont\small}
\linenumbers


% COUNTERS
%------------------------------------------
% TRB requires the total number of words, figures, and tables to be displayed on
% the title page. This is possible under the totcount package on CTAN.
\usepackage{totcount}
	\regtotcounter{table} 	%count tables
	\regtotcounter{figure} 	%count figures

\newcommand\wordcount{\immediate\write18{texcount -sum -1 \jobname.tex > 'count.txt'} \input{count.txt} }
  
% DOCUMENT START
%------------------------------------------
% Add any additional \usepackage declarations here.

\usepackage{graphicx}
\graphicspath{{./images/}}
 
\usepackage{rotating}
\usepackage{longtable}

%\usepackage{enumerate}
\usepackage{paralist}

%%%TIKZ
\usepackage{tikz}
\usepackage{pgfplots}
\usepackage{pgfplotstable}
\usepackage{pgfgantt}
\pgfplotsset{compat=newest}

\usetikzlibrary{arrows,shapes,positioning,shapes.geometric}
\usetikzlibrary{decorations.markings}
\usetikzlibrary{shadows,automata}
\usetikzlibrary{patterns}
\usetikzlibrary{trees,mindmap,backgrounds}
%\usetikzlibrary{circuits.ee.IEC}
\usetikzlibrary{decorations.text}
% For Sagnac Picture
\usetikzlibrary{%
    decorations.pathreplacing,%
    decorations.pathmorphing%
}
\tikzset{no shadows/.style={general shadow/.style=}}
%
%\usepackage{paralist}

%%% FORMAT PYTHON CODE
\usepackage{listings}
% Default fixed font does not support bold face
\DeclareFixedFont{\ttb}{T1}{txtt}{bx}{n}{8} % for bold
\DeclareFixedFont{\ttm}{T1}{txtt}{m}{n}{8}  % for normal

% Custom colors
\usepackage{color}
\definecolor{deepblue}{rgb}{0,0,0.5}
\definecolor{deepred}{rgb}{0.6,0,0}
\definecolor{deepgreen}{rgb}{0,0.5,0}

\newcommand{\osn}{\oldstylenums}
\newcommand{\dg}{^{\circ}}
\newcommand{\lt}{\left}
\newcommand{\rt}{\right}
\newcommand{\pt}{\phantom}
\newcommand{\tf}{\therefore}
\newcommand{\?}{\stackrel{?}{=}}
\newcommand{\fr}{\frac}
\newcommand{\dfr}{\dfrac}
\newcommand{\ul}{\underline}
\newcommand{\tn}{\tabularnewline}
\newcommand{\nl}{\newline}
\newcommand\relph[1]{\mathrel{\phantom{#1}}}
\newcommand{\cm}{\checkmark}
\newcommand{\ol}{\overline}
\newcommand{\rd}{\color{red}}
\newcommand{\bl}{\color{blue}}
\newcommand{\pl}{\color{purple}}
\newcommand{\og}{\color{orange!90!black}}
\newcommand{\gr}{\color{green!40!black}}
\newcommand{\nin}{\noindent}
\newcommand{\la}{\lambda}
\renewcommand{\th}{\theta}
\newcommand{\al}{\alpha}
\newcommand{\G}{\Gamma}
\newcommand*\circled[1]{\tikz[baseline=(char.base)]{
            \node[shape=circle,draw,thick,inner sep=1pt] (char) {\small #1};}}

\newcommand{\bc}{\begin{compactenum}[\quad--]}
\newcommand{\ec}{\end{compactenum}}

\newcommand{\p}{\partial}
\newcommand{\pd}[2]{\frac{\partial{#1}}{\partial{#2}}}
\newcommand{\dpd}[2]{\dfrac{\partial{#1}}{\partial{#2}}}
\newcommand{\pdd}[2]{\frac{\partial^2{#1}}{\partial{#2}^2}}

\begin{document}
\renewcommand{\refname}{\uppercase{References}}
% \raggedright

\begin{titlepage}
\begin{flushleft}

% Title
{\LARGE \bfseries Impacts of COVID-19 on urban rail transit energy consumption: A case study of Boston}\\[1cm]

%%%%%%%%%%%%%%%%%%%%%%%%%%%%%%
%%% Provisional author list; to be finalized once draft is ready.
%%%%%%%%%%%%%%%%%%%%%%%%%%%%%%5

 \textbf{Zhuo Han\textsuperscript{1}} \\
 Email: zhuohan@umass.edu  \\[0.5cm]

 \textbf{Jimi B.\ Oke\textsuperscript{1}} \\
 %\textit{Corresponding Author} \\
 Email: jboke@umass.edu  \\[0.5 cm]

% \textbf{Author 3 \textsuperscript{1}} \\
% Email: @ ; Phone: \\[0.5 cm]
 
 
\textbf{\textsuperscript{1} Department of Civil and Environmental Engineering, University of Massachusetts Amherst, MA 01003, United States} \\[0.3 cm]
%\textbf{\textsuperscript{2} Department of Civil and Environmental Engineering, University of Massachusetts Amherst, MA 01003, United States} \\[0.3 cm]


\wordcount words + \total{table} tables %+ \total{figure} figures

\today

\end{flushleft}
\end{titlepage}

%% This generates the title page from the information given above.
\thispagestyle{empty}
%\maketitle

\newpage

\thispagestyle{empty}
 

\section{Abstract}


\section{Introduction}



\section{Data summary}

The variables are summarized in \autoref{tab:variables}.
\begin{table}[ht]\footnotesize
    \centering
    \begin{tabular}{l l}\toprule
        \bf Variable & \bf Description \\
       speed  &  train speed at each time record \\
       acceleration  &  train acceleration at each time record \\
       temperature & daily temperature in Boston area\\
       ridership & tap-in ridership data for each half hour\\
       distance & vehicle distance between each two time records\\
       time\_hr & train operating time on each hour\\
       average hourly speed & the average speed in one hour\\
       speed\_X\_bin & time spent in an equal-probability speed interval \\
       acceleration \_X\_bin & time spent in an equal-probability acceleration interval \\
       \bottomrule
    \end{tabular}
    \caption{Summary of variables used in model}
    \label{tab:variables}
\end{table}




\section{Methodologies}


\subsection{Multiple linear regression}



\subsection{COVID impacts analysis}



\section{Results and Discussion}



\section{Conclusion and future work}

 
% \section{Author Contribution Statement}
% The authors confirm contribution to the paper as follows: \\
% Study conception and design: \\
% Analysis and interpretation of results: \\
% Manuscript preparation:

%\section{Acknowledgment}

\bibliography{references.bib}
\bibliographystyle{trb}


\end{document}

%%% Local Variables:
%%% mode: latex
%%% TeX-master: t
%%% End:

